\declareIM{ppf}{pre-pre-F}{2021-06-04}{AOS using Corner raft}{AOS Corner}
\completeIM{\thisIM}{2021-09-01 \JIRA{DM}{30764}{}}

Executive Summary:
​
Demonstrate the processing of wave front images for raw corner raft \texttt{SW0/1} simulated data
to Zernikes.

\textbf{Does not map to P6}
​
\subsection{Goals of IM}
​
Demonstrate ability to:
​
\begin{itemize}
\item Process corner raft Images and Generate Zernike coefficients
\end{itemize}
​
\subsection{Prerequisites}
\begin{itemize}
\item PhoSim images with donuts from \texttt{SW0/1} ingested into a butler repo.
\item Working version of wavefront estimation pipeline (wep). 
\end{itemize}

\subsection{Procedure}
\begin{itemize}
\item Generate and ingest simulated image and appropriate metadata.
  \begin{itemize}
  \item Boresight and rotator angle
  \item Piston from camera hexapod
  \end{itemize}
\item Run a Gen3 pipeline from the command line that:
  \begin{itemize}
  \item gets data from the butler
  \item runs the ISR
  \item finds and process isolated donuts
  \item average the resulting Zernikes over the stars in the field.
  \end{itemize}
  \item uses the butler to put the description of the wavefront to disk
\end{itemize}

\subsection{Acceptance Criteria}
\begin{itemize}
\item A member of SITCom must be able to carry out these operations at NCSA.
  This need not be done using RSP; a login shell on \eg \texttt{lsst-devl3} is acceptable, and
  the SITCom member may be required to install and build packages from \texttt{git}.
\item Confirm that the results are as expected.  This should be carried out using a notebook on the RSP,
  and the tester may be required to install the notebook from \texttt{git}.
\end{itemize}
