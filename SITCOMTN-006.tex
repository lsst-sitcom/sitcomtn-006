\documentclass[SE,authoryear,toc]{lsstdoc}
% lsstdoc documentation: https://lsst-texmf.lsst.io/lsstdoc.html
\input{meta}

% Package imports go here.
\usepackage{xspace}

% Local commands go here.
\renewcommand{\c}{\textit{c.}\xspace}
\newcommand{\cf}{\textit{cf.}\xspace}

\newcommand{\IM}[1]{\hyperref[IM:#1]{\color{blue}IM$_{\text{\csname nameIM#1\endcsname}}$}}
\newcommand{\IMdate}[1]{\csname dateIM#1\endcsname}
\makeatletter
\newcommand{\declareIM}[4]{%
  % \declareIM{label}{short name}{date}{title}
  % e.g.
  %    \declareIM{b}{B}{2021-XX-YY}{ComCam Image Capture and Archive}
  % where you refer to the IM as \IM{b} and it appears as IM_B
  %
  \section{IM\texorpdfstring{$_{\text{#2}}$}{#2} [#1]: #4 (#3)}%
  \label{IM:#1}%
  \write\@auxout{\unexpanded{\expandafter\gdef\csname nameIM#1\endcsname}{#2}}%
  \write\@auxout{\unexpanded{\expandafter\gdef\csname dateIM#1\endcsname}{#3}}%
}
\makeatother
\newcommand{\JIRA}[3]{\href{https://jira.lsstcorp.org/browse/#1-#2}{(\textcolor{blue}{#1-#2} #3})}

%If you want glossaries
\makeglossaries
% DO NOT EDIT - generated by /Users/rhl/LSST/texmf/bin/generateAcronyms.py from https://lsst-texmf.lsst.io/.
\newglossaryentry{} {name={}, description={}}
\newacronym{AURA} {AURA} {\gls{Association of Universities for Research in Astronomy}}
\newglossaryentry{Association of Universities for Research in Astronomy} {name={Association of Universities for Research in Astronomy}, description={ consortium of US institutions and international affiliates that operates world-class astronomical observatories, AURA is the legal entity responsible for managing what it calls independent operating Centers, including LSST, under respective cooperative agreements with the National Science Foundation. AURA assumes fiducial responsibility for the funds provided through those cooperative agreements. AURA also is the legal owner of the AURA Observatory properties in Chile}}
\newglossaryentry{Butler} {name={Butler}, description={A middleware component for persisting and retrieving image datasets (raw or processed), calibration reference data, and catalogs}}
\newacronym{CCS} {CCS} {Camera Control System}
\newglossaryentry{Camera} {name={Camera}, description={The LSST subsystem responsible for the 3.2-gigapixel LSST camera, which will take more than 800 panoramic images of the sky every night. SLAC leads a consortium of Department of Energy laboratories to design and build the camera sensors, optics, electronics, cryostat, filters and filter exchange mechanism, and camera control system}}
\newglossaryentry{Center} {name={Center}, description={An entity managed by AURA that is responsible for execution of a federally funded project}}
\newglossaryentry{Construction} {name={Construction}, description={The period during which LSST observatory facilities, components, hardware, and software are built, tested, integrated, and commissioned. Construction follows design and development and precedes operations. The LSST construction phase is funded through the \gls{NSF} \gls{MREFC} account}}
\newacronym{DM} {DM} {\gls{Data Management}}
\newacronym{DMS} {DMS} {Data Management Subsystem}
\newacronym{DMTN} {DMTN} {DM Technical Note}
\newacronym{DOE} {DOE} {\gls{Department of Energy}}
\newglossaryentry{Data Management} {name={Data Management}, description={The LSST Subsystem responsible for the Data Management System (DMS), which will capture, store, catalog, and serve the LSST dataset to the scientific community and public. The DM team is responsible for the DMS architecture, applications, middleware, infrastructure, algorithms, and Observatory Network Design. DM is a distributed team working at LSST and partner institutions, with the DM Subsystem Manager located at LSST headquarters in Tucson}}
\newglossaryentry{Data Management Subsystem} {name={Data Management Subsystem}, description={The Data Management Subsystem is one of the four subsystems which constitute the LSST Construction Project. The Data Management Subsystem is responsible for developing and delivering the LSST Data Management System to the LSST Operations Project}}
\newglossaryentry{Data Management System} {name={Data Management System}, description={The computing infrastructure, middleware, and applications that process, store, and enable information extraction from the LSST dataset; the DMS will process peta-scale data volume, convert raw images into a faithful representation of the universe, and archive the results in a useful form. The infrastructure layer consists of the computing, storage, networking hardware, and system software. The middleware layer handles distributed processing, data access, user interface, and system operations services. The applications layer includes the data pipelines and the science data archives' products and services}}
\newglossaryentry{Department of Energy} {name={Department of Energy}, description={cabinet department of the United States federal government; the DOE has assumed technical and financial responsibility for providing the LSST camera. The DOE's responsibilities are executed by a collaboration led by SLAC National Accelerator Laboratory}}
\newglossaryentry{DocuShare} {name={DocuShare}, description={The trade name for the enterprise management software used by LSST to archive and manage documents}}
\newglossaryentry{Document} {name={Document}, description={Any object (in any application supported by DocuShare or design archives such as PDMWorks or GIT) that supports project management or records milestones and deliverables of the LSST Project}}
\newacronym{EFD} {EFD} {Engineering and Facility Database}
\newacronym{EPO} {EPO} {Education and Public Outreach}
\newglossaryentry{Education and Public Outreach} {name={Education and Public Outreach}, description={The LSST subsystem responsible for the cyberinfrastructure, user interfaces, and outreach programs necessary to connect educators, planetaria, citizen scientists, amateur astronomers, and the general public to the transformative LSST dataset}}
\newglossaryentry{Handle} {name={Handle}, description={The unique identifier assigned to a document uploaded to DocuShare}}
\newacronym{IM} {IM} {Integration Milestone}
\newacronym{LDM} {LDM} {LSST Data Management (Document Handle)}
\newacronym{LOVE} {LOVE} {LSST Operations Visualization Environment}
\newacronym{LSST} {LSST} {Legacy Survey of Space and Time (formerly Large Synoptic Survey Telescope)}
\newacronym{MREFC} {MREFC} {\gls{Major Research Equipment and Facility Construction}}
\newglossaryentry{Major Research Equipment and Facility Construction} {name={Major Research Equipment and Facility Construction}, description={the NSF account through which large facilities construction projects such as LSST are funded}}
\newacronym{NCSA} {NCSA} {National Center for Supercomputing Applications}
\newacronym{NOAO} {NOAO} {National Optical Astronomy Observatories (USA)}
\newacronym{NSF} {NSF} {\gls{National Science Foundation}}
\newglossaryentry{National Science Foundation} {name={National Science Foundation}, description={primary federal agency supporting research in all fields of fundamental science and engineering; NSF selects and funds projects through competitive, merit-based review}}
\newacronym{OCS} {OCS} {Observatory Control System}
\newacronym{OR} {OR} {Operation Rehearsal}
\newacronym{ORR} {ORR} {Operations Readiness Review}
\newglossaryentry{Operations} {name={Operations}, description={The 10-year period following construction and commissioning during which the LSST Observatory conducts its survey}}
\newglossaryentry{Project Manager} {name={Project Manager}, description={The person responsible for exercising leadership and oversight over the entire LSST project; he or she controls schedule, budget, and all contingency funds}}
\newacronym{QA} {QA} {Quality Assurance}
\newacronym{QC} {QC} {Quality Control}
\newglossaryentry{Quality Assurance} {name={Quality Assurance}, description={All activities, deliverables, services, documents, procedures or artifacts which are designed to ensure the quality of DM deliverables. This may include \gls{QC} systems, in so far as they are covered in the charge described in LDM-622. Note that contrasts with the LDM-522 definition of “QA” as “Quality Analysis”, a manual process which occurs only during commissioning and operations. See also: Quality Control}}
\newglossaryentry{Quality Control} {name={Quality Control}, description={Services and processes which are aimed at measuring and monitoring a system to verify and characterize its performance (as in LDM-522). Quality Control systems run autonomously, only notifying people when an anomaly has been detected. See also Quality Assurance}}
\newacronym{RAT} {RAT} {Rubin Auxiliary Telescope}
\newglossaryentry{RSP} {name={RSP}, description={Rubin Science Platform}}
\newglossaryentry{Review} {name={Review}, description={Programmatic and/or technical audits of a given component of the project, where a preferably independent committee advises further project decisions, based on the current status and their evaluation of it. The reviews assess technical performance and maturity, as well as the compliance of the design and end product with the stated requirements and interfaces}}
\newacronym{SE} {SE} {System Engineering}
\newacronym{SLAC} {SLAC} {SLAC National Accelerator Laboratory (formerly Stanford Linear Accelerator Center; SLAC is now no longer an acronym)}
\newglossaryentry{Science Platform} {name={Science Platform}, description={A set of integrated web applications and services deployed at the LSST Data Access Centers (DACs) through which the scientific community will access, visualize, and perform next-to-the-data analysis of the LSST data products}}
\newglossaryentry{Subsystem} {name={Subsystem}, description={A set of elements comprising a system within the larger LSST system that is responsible for a key technical deliverable of the project}}
\newglossaryentry{Subsystem Manager} {name={Subsystem Manager}, description={responsible manager for an LSST subsystem; he or she exercises authority, within prescribed limits and under scrutiny of the Project Manager, over the relevant subsystem's cost, schedule, and work plans}}
\newacronym{T&S} {T&S} {Telescope and Site}
\newacronym{TBD} {TBD} {To Be Defined (Determined)}
\newglossaryentry{Telescope and Site} {name={Telescope and Site}, description={The LSST subsystem responsible for design and construction of the telescope structure, telescope mirrors, optical wavefront measurement and control system, telescope and observatory control systems software, and the summit and base facilities. The Telescope technical team is hosted by NOAO}}
\newacronym{US} {US} {United States}
\newglossaryentry{calibration} {name={calibration}, description={The process of translating signals produced by a measuring instrument such as a telescope and camera into physical units such as flux, which are used for scientific analysis. Calibration removes most of the contributions to the signal from environmental and instrumental factors, such that only the astronomical component remains}}
\newglossaryentry{camera} {name={camera}, description={An imaging device mounted at a telescope focal plane, composed of optics, a shutter, a set of filters, and one or more sensors arranged in a focal plane array}}
\newglossaryentry{configuration} {name={configuration}, description={A task-specific set of configuration parameters, also called a 'config'. The config is read-only; once a task is constructed, the same configuration will be used to process all data. This makes the data processing more predictable: it does not depend on the order in which items of data are processed. This is distinct from arguments or options, which are allowed to vary from one task invocation to the next}}
\newglossaryentry{flux} {name={flux}, description={Shorthand for radiative flux, it is a measure of the transport of radiant energy per unit area per unit time. In astronomy this is usually expressed in cgs units: erg/cm2/s}}
\newglossaryentry{middleware} {name={middleware}, description={Software that acts as a bridge between other systems or software usually a database or network. Specifically in the Data Management System this refers to Butler for data access and Workflow management for distributed processing.}}
\newglossaryentry{monitoring} {name={monitoring}, description={In DM QA, this refers to the process of collecting, storing, aggregating and visualizing metrics}}

\glsunsetall  % never expand acronyms
\iftrue\else % colour rather than box acronyms
   \AtBeginDocument{\hypersetup{pdfborder={0 0 0}}}
   \renewcommand*{\glstextformat}[1]{\textcolor{red}{#1}}
\fi

\title{Integration Milestones}

% Optional subtitle
% \setDocSubtitle{A subtitle}

\author{%
Robert Lupton
}

\setDocRef{SITCOMTN-006}
\setDocUpstreamLocation{\url{https://github.com/lsst-sitcom/sitcomtn-006}}

\date{\vcsDate}

% Optional: name of the document's curator
% \setDocCurator{The Curator of this Document}

\setDocAbstract{%
A proposal for ways to work on predominantly software subsystem commissioning,  building on the Rubin AuxTel (n\'ee auxTel) experience
}

% Change history defined here.
% Order: oldest first.
% Fields: VERSION, DATE, DESCRIPTION, OWNER NAME.
% See LPM-51 for version number policy.
\setDocChangeRecord{%
  \addtohist{1}{YYYY-MM-DD}{Unreleased.}{Robert Lupton}
}

\begin{document}

\maketitle

\section{Introduction}

The Rubin construction project is composed of three main subsystems: DM, Camera, and T\&S.\footnote{
  And also EPO, but this is probably not relevant here.}
but in operations there is only one survey.  The task of merging the projects is not easy, especially
when some of the interfaces are underspecified, or turn out to need modification based on experience.
This is primarily
a problem for the software --- the hardware interfaces, specified in mm and screw threads, are generally
well defined.

Once we accept that the only way to complete the integration of the subsystems is to actually integrate them,
while expecting that the interfaces and requirements will evolve, we recognise that our
situation is analogous to the
way that software development has moved from waterfall to agile development models.

This technote proposes that we introduce the concept of an \gls{IM} which defines
a specified set of functionalities that are available to the Rubin project.
This is slightly different from \gls{DM}'s
current \gls{OR} concept which tests the state of DM systems at a certain time rather than guaranteeing
continued cross-subsystem functionality.  I envisage that an \gls{IM} would be followed by an \gls{OR} to test
the deliverable, and to carry out an internal \gls{ORR} for the functionality delivered by the \gls{IM}.

\subsection{Dependencies}

\begin{figure}
\begin{center}
  %\rotatebox{90}{\includegraphics[width=1.1\textwidth]{dependencies}}
  \hspace*{-0.15\textwidth}\includegraphics[width=1.3\textwidth]{dependencies}
\end{center}
\caption{
  The dependencies between the IMs and other entities, either external hardware or software
  (\eg ComCam, Nublado shown in hexagons), or systems being produced during the completion
  of the Rubin system (\eg the \gls{OCPS}, shown in ellipses and whose first appearance
  is shown as a green line).
  The rectangular boxes are IMs; red means complete.  The blue arrows show IM's prerequisites,
  and black arrows indicate other dependencies.
  \hfil\break
  Not all IMs are shown at this point.  The mapping to P6 is also omitted as it is not
  yet clear enough to be useful (but see App. \ref{sec:P6Milestones}).
}
\label{fig:dependencies}
\end{figure}

% See Fig. \ref{fig:dependencies}.

\section{Schedule}

\begin{center}
  \begin{tabular}{l|c|l}
    IM & Completion Date & Comments \\
    \hline
    \IM{a} & \IMdate{a} \\
    \IM{b} & \IMdate{b} & OCPS will be tight; slip logging \\
    \IM{c} & \IMdate{c} & OCPS will be tight; slip logging \\
    \IM{d} & \IMdate{d} & \\
    \IM{e} & \IMdate{e} & \\
    \IM{f} & \IMdate{f} & In parallel with \IM{pg} \\
    \IM{ppg} & \IMdate{ppg} & In parallel with \IM{f} \\
    \IM{pg} & \IMdate{pg} & In parallel with \IM{f} \\
    \IM{g} & \IMdate{g} \\
    \IM{h} & \IMdate{h} \\
  \end{tabular}
\end{center}

Those are pretty aggressive dates, but they are completed \textit{just} before we 
put ComCam on the TMA (without mirrors).

\newcommand{\inputIM}[1]{\vfil\eject\input{#1}}
\inputIM{IMa}
\inputIM{IMb}
\inputIM{IMc}
\inputIM{IMd}
\inputIM{IMe}
\inputIM{IMf}
\inputIM{IMppg}
\inputIM{IMpg}
\inputIM{IMg}
\inputIM{IMh}
\inputIM{IMi}
\inputIM{IMj}
\inputIM{IMk}
\inputIM{IMcbp}  % == L
\inputIM{IMbacklog}

\appendix

\section{Mapping to Milestones in P6}
\label{sec:P6Milestones}

This list of project milestones comes from
\href{https://confluence.lsstcorp.org/pages/viewpage.action?pageId=62263966}{this page}
on confluence; the descriptions are summarised from the same page.  We omit
\textit{Integrate ComCam on Hexapod-Rotator and Install on TMA},
\textit{Technical Operations Optimization 1-3},
and
\textit{Engineering Punch-list Resolution} milestones.

\subsection{on-Telescope ComCam TCS + CCS Interface \& Functional Tests}

\begin{itemize}
\item
  TCS functional testing of ComCam configuration (Load in guider and WFS applications, check high level functionality/operation without data acquisition )
\item
  CCS functional testing of ComCam - configuration update, shutter testing, filter changer, temperature control, data acquisition, data display, setup of trending analysis. Most of these have to be done as a function of telescope position
\end{itemize}
Prerequisites: \IM{b}, \IM{g}

\subsection{on-Telescope ComCam CCS + OCS Interface \& Functional Tests}

\begin{itemize}
\item LSE-71 verification - (filter, shutter, image acquisition commands etc)
\item Observatory mode testing (e.g. calibration, daytime, nighttime ops, engineering mode)
\item CCS communication with EFD and observatory clock
\item OCS Sequencer testing (e.g. take flat field sequence)
\end{itemize}
Prerequisites: \IM{b}

\subsection{on-Telescope ComCam CCS + DAQ + DMS Interface \& Functional Tests}
\begin{itemize}
\item Take image (start with a bias) and send through DMS
\item Test on-summit data access portal, Commissioning Cluster data access
\item Test archiving speeds, network speeds/lags, start long-term monitoring
\item Test header service
\item Test EFD access
\item Repeat for all normal exposure types (broadband flats, monochromatic flats, darks, dome-closed "on-sky" image, CBP image
\item Attempt ingestion into pipelines to verify header service/EFD access (captured in CPP activity)
\item Test base-facility connection loss/recovery
\item Test base-facility to NCSA connection loss/recovery
\item Test grid-power loss
\item A stretched goal is to exercise the interface between the DAQ and potentially both the guider and the wfs
\end{itemize}
Prerequisites: \IM{b}, \IM{cbp}.  \IM{g} for stretch goal.

\subsection{ComCam Electro-Optical Tests 1}

\begin{itemize}
\item Master Bias Creation - 30 minutes of data collection ($\sim$100 images at 15s cadence)
\item Gain value verification - ?do on 3 filters once? - assume half a day - flat field Photon transfer curve
\item Linearity Verification - same data set as above
\item Saturation Determination () - same as above - but doesn't infer when bleeding occurs
\item Master Dark creation () - 10x300s, 10x30s, 10x60s,10x120s, 10x600s = 11100s = 3 hours
\item Master Impure Broadband Flats () - 1 hour for 5/6 filters
\item Master Impure Monochromatic Flats (4 hours per filter + 8 hours for no filter (assume 2 hours but needs to be done $\sim$ 4 times depending on separation between filter tests)
\end{itemize}
Prerequisites: \IM{b}.

\subsection{Collimated Beam Projector Scripting Tests}

\begin{itemize}
\item Image Quality Checks as a function of telescope Elevation to validate FEA models
\item short un-guided images
\item Analysis of data
\item Develop automated pointing model builder
\item Point Model Building - 50 iterations
\item Acquire image of known star
\item Centroid and offset to pixel X,Y
\item Verify star position at pixel X,Y
\item Move to next target
\item Do touch ups over multiple nights ($\sim$1 hour per night for 5 nights)
\end{itemize}
Prerequisites: \IM{cbp}.

\subsection{Initial OCS + TCS + CCS Guider Interface \& Functional Tests}

Initial TCS + DAQ WFS Interface Tests
\begin{itemize}
\item obtain in+intra+extra focal images and feed through wavefront analysis system
\item generate AOS offsets, correlate with observing conditions (e.g. atl/az, temp, hum)
\item update look up tables, re-check
\item verify offsets are being propogated properly (purposely deform and measure effect)
\item attempt stress the system over wide range of conditions
\end{itemize}
Prerequisites: pointing model, \IM{b}, \IM{g}
  
\subsection{ComCam Electro-Optical Tests 2}

\begin{itemize}
\item ghosties and ghoulies
\item Master Impure and Pure "Monochromatic" Flats
\item Master Photoflats, Master Low-resolution narrowband Flats
\item strip-chart imaging to measure slew/settle motions/times.
\item Try scheduled driven observing with ComCam - digests telemetry, feeds targets, logs observations
\end{itemize}
Prerequisites: \IM{b}, \IM{g}

\subsection{Build Initial AOS Look-up Table}

The Look-up-table has been verified on axis during T\&S AIV with the high speed camera. This test extend the verification of the lookup table to the wider ComCam field of view

\begin{itemize}
\item Conduct the initial alignment using the laser tracker
\item Execute script driven sampling of telescope optical deformation by measuring IQ as a function of elevation (and azimuth). The use case is described in the use case document (here)
\item Obtain force offsets to adjust optics for each elevation (and azimugh)
\item Update/build look up table as a function of elevation (and azimuth)
\end{itemize}
Prerequisites: \IM{b}, \IM{g}

\subsection{AOS Data Analysis 1}

\begin{itemize}
\item Track IQ, wavefront and AOS telemetry over history to date (1-2 weeks)
\item update look up tables expanding to environmental conditions (temp, hum, wind)
  This is done in parallel with the task above
\end{itemize}
Prerequisites: \IM{b}

\subsection{ComCam Electro-Optical Tests 3}

\begin{itemize}
\item Combined testing of on-sky data with AT and ComCam - with supporting telemetry (all-sky cameras, dimm, PWV etc)
\item Try scheduled driven observing with ComCam (excercising level 2)
\item Comparison of star flat to CBP flat, ghost characterization investigation, filter QA system setup/measurement
\end{itemize}
Prerequisites: \IM{b}, \IM{g}, \IM{cbp}

\subsection{Initial AOS Performance Verification}

\begin{itemize}
\item track IQ, wavefront and AOS telemetry over history to date ($\sim$1 month)
\item update look up tables
\end{itemize}
Prerequisites: \IM{b}

% Include all the relevant bib files.
% https://lsst-texmf.lsst.io/lsstdoc.html#bibliographies
\section{References} \label{sec:bib}
\renewcommand{\refname}{} % Suppress default Bibliography section
\bibliography{local,lsst,lsst-dm,refs_ads,refs,books}

% Make sure lsst-texmf/bin/generateAcronyms.py is in your path
%\section{Acronyms} \label{sec:acronyms}
%\addtocounter{table}{-1}
\begin{longtable}{p{0.145\textwidth}p{0.8\textwidth}}\hline
\textbf{Acronym} & \textbf{Description}  \\\hline

 &  \\\hline
AURA & Association of Universities for Research in Astronomy \\\hline
CCS & Camera Control System \\\hline
DM & Data Management \\\hline
DMS & Data Management Subsystem \\\hline
DMTN & DM Technical Note \\\hline
DOE & Department of Energy \\\hline
EFD & Engineering and Facility Database \\\hline
EPO & Education and Public Outreach \\\hline
IM & Integration Milestone \\\hline
LDM & LSST Data Management (Document Handle) \\\hline
LOVE & LSST Operations Visualization Environment \\\hline
LSST & Legacy Survey of Space and Time (formerly Large Synoptic Survey Telescope) \\\hline
MREFC & Major Research Equipment and Facility Construction \\\hline
NCSA & National Center for Supercomputing Applications \\\hline
NOAO & National Optical Astronomy Observatories (USA) \\\hline
NSF & National Science Foundation \\\hline
OCS & Observatory Control System \\\hline
OR & Operation Rehearsal \\\hline
ORR & Operations Readiness Review \\\hline
QA & Quality Assurance \\\hline
QC & Quality Control \\\hline
RAT & Rubin Auxiliary Telescope \\\hline
RSP & Rubin Science Platform \\\hline
SE & System Engineering \\\hline
SLAC & SLAC National Accelerator Laboratory (formerly Stanford Linear Accelerator Center; SLAC is now no longer an acronym) \\\hline
T\&S & Telescope and Site \\\hline
TBD & To Be Defined (Determined) \\\hline
US & United States \\\hline
\end{longtable}

% If you want glossary uncomment below -- comment out the two lines above
\printglossaries

\end{document}
