\declareIM{b}{2}{2021-05-15}{ComCam Calibration Acquisition, Transfer, and Processing}

Executive Summary:

Run ComCam from notebooks and the scriptQueue, with automated generation and certification of ComCam calibrations

\textbf{Does not map well. Somewhat maps to same milestones as previous exercise, but no solid mapping as this is more about operator awareness and DM related functionality. }

\textbf{\Nb this is the IM that mentions the use of the EFD.  We should expand this, and maybe split it out.}

\subsection{Goals of IM}
\label{sec:org3154347}
\begin{itemize}
	\item Triggering \gls{OCPS} processing from nublado/scriptQueue
	\item Automated generation and availability of ComCam master calibrations in Chile
	\item Use of \gls{RLS} and \gls{LOVE} to manage scriptQueue
	\begin{itemize}
		\item including monitoring of data transfer to NCSA
	\end{itemize}
	\item Demonstration that the operator can cycle through states, bring
	the system up from \texttt{STANDBY}, and take data without any intervention from developers
	\item Monitor the health of the \gls{CCS} from the observatory environment (e.g. Chronograph/\gls{LOVE})
	\item Explore relationship between \gls{EFD} and Camera trending databases
	\item Query capability (TAP/ADQL) access to \gls{EFD} from \gls{RSP}
\end{itemize}

\subsection{Prerequisites}
\begin{itemize}
	\item{\IM{a} completed with functionality maintained}
	\item{ComCam cold and functional with light source available for flat}
\end{itemize}


\subsection{Procedure}
The following procedure is to be executed by an observing specialist. Assistance may be provided in the script editing and database querying by other commissioning team members.
\begin{itemize}
	\item Following a procedure, bring up the LOVE interface for ComCam
	\item Using Nublado, without assistance from developers, the operator should bring ComCam to ENABLED from STANDBY
	\item Take an OBJECT image using the \href{https://ts-observatory-control.lsst.io/py-api/lsst.ts.observatory.control.maintel.ComCam.html}{ComCam class}
	\begin{itemize}
		\item Verify functionality of the \gls{LOVE} interface, including the state and health
		\item Display the raw image locally
		\item Using the OCPS, perform basic \gls{ISR} on the image and wait for the result (e.g. "await ocps.process(visitID, task\_name)")
		\item Display the ISR-corrected image
		\item Monitor the transfer status and elapsed time between when the image is written on the summit to when it can be accessed at NCSA.
	\end{itemize}	
	\item Modify the script from \IM{a} used to take calibrations to also command the \gls{OCPS} to reduce the data
	\begin{itemize}
		\item Take \emph{only} a stack of $\sim$10 bias frames
		\item Using the \gls{OCPS}, build a master bias from inside the script. Do not wait for the processing to finish 
			\begin{itemize}
				\item One could use "asyncio.future(ocps.process(visitID, task\_name))"
			\end{itemize}
		\item Ensure the bias builds and is certified % FIXME: HOW do we do this?
	\end{itemize}
	\item From the \gls{RSP}, query the local copy of the EFD and plot exposure time versus number since the start of this exercise
	\item From the RSP, make the same plot using data from the Camera Trending Database
	% FIXME: The above item requires futher detailing on how we exercise the CTD
\end{itemize}

