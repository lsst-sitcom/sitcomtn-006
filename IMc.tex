\declareIM{c}{C}{2021-XX-YY}{Rubin AuxTel Standard Star Campaign}

Executive Summary:

Demonstrates ability to take and reduce AuxTel data using the scriptQueue.

\textbf{No direct mapping to P6 activity, indirect mappings are below}

\subsection{Goals of IM}
\begin{itemize}
\item Automated generation and availability of AuxTel master calibrations in Chile (cf. \IM{b}) - \textbf{SUMMIT-2989 Atmospheric Calibration Products Pipeline Testing}
\item Annotate exposures using \gls{RLS}
\item Take data in a standard star field, including interactive re-centering \textbf{}
\item Run AuxTel data analysis package in Chile and NCSA, triggered by \gls{OCPS} \textbf{Partial mapping to SUMMIT-2987 - Calibration Telescope Instrumental Signature Removal Testing \& Verification and SUMMIT-2871 - Aux. Tel. DAQ + DMS Interface Testing}
\begin{itemize}
\item Display results to operators % (may require publishing to EFD?)
\end{itemize}
\end{itemize}

\subsection{Prerequisites}
\begin{itemize}
	\item{\IM{b} Completed}
	\item{AuxTel and LATISS functional and ready to track a target. Rough focus should be achieved.}
	\item{Available staff for night time operations}
	\item{User commenting in \gls{RLS} available}
\end{itemize}

\subsection{Procedure}
The following procedure is to be executed by an observing specialist.
\begin{enumerate}
	\item Using Nublado, slew to $\sim$~60 degree elevation target and start tracking
	\item Using Nublado, measure the focus offset using the \href{https://github.com/lsst-ts/ts_externalscripts/blob/develop/python/lsst/ts/externalscripts/auxtel/latiss_cwfs_align.py}{CWFS focus script}, but launched from the notebook.
	\item Annotate results using the \gls{RLS}
	\item Apply the focus offset using the ATAOS, record the applied offset in the \gls{RLS}
	\item Manually take a single OBJECT image:
	\begin{enumerate}
		\item Verify functionality of the \gls{LOVE} interface, including instrument setup and applied focus offset(s).
		\item Using the \gls{OCPS}, perform basic \gls{ISR} on the image and wait for the result (e.g. "await ocps.process(visitID, task\_name)")
		\item Display the image locally after running \gls{ISR}
	\end{enumerate}	
	\item Using the standard visit script launched from a notebook, slew to new target and perform automated acquisition only, script must have user verify offset prior to motion.
	\item Perform the standard data taking sequence using the standard visit script
	\item Perform a single image with the instrument in it's standard spectral mode, then send for reduction with the \gls{OCPS}. Wait for reduction to complete.
	\item View the results locally in the notebook.
	\item Perform a custom sequence using at least two items in the grating wheel and two in the filter wheel, all with different exposure times (\gls{OCPS} should only perform ISR)
	\begin{enumerate}
		\item Verify functionality of the \gls{LOVE} interface, including filter/grating changes/shutter movement, focus/pointing offsets, \gls{OCPS} status 
		% Do we get any OCPS status? How do we monitor that?
		\item Display the \gls{ISR} processed images locally as they are available
		\item Monitor the transfer status and elapsed time between when the image is written on the summit to when it can be accessed from the \gls{RSP}.
	\end{enumerate}	
	\item Using the scriptQueue:
		\begin{enumerate}
			\item Perform standard observations of 3 targets spanning the elevation range sequentially, script must send data to \gls{OCPS} for reduction (but not await results)
			\item Display extracted atmospheric parameters for each target % HOW?
			\item Comment in \gls{RLS} about the weather or image quality
			\item Perform same observations, but with a custom modified configuration (exposure time or filter adjustment)
		\end{enumerate}
	\item From the Base and the \gls{RSP}, redo reduction of scriptQueue visit data
	\item From the Base and the \gls{RSP}, view the log and correlate comments against images.
\end{enumerate}